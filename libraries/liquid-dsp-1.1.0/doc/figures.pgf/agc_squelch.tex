% 
% figures.pgf/agc_squelch.tex
%
% Author: Joseph D. Gaeddert
%
%              _____________________
%             /                     \
%            /                       \
%           /                         \
%          /                           \
%         /                             \
%     ---X--- threshold      ...      ---X---
%       /                                 \
% _____/___       ...     noise floor______\_____|_____
%
% ta   tb     tc                    td      te          tf
% |    |      |                     |       |           |
% +----+-+----+---------------------+----+--+----+------+----> time
%        |                               |       |
%        rise                            fall    timeout
%        

\documentclass[a4paper,landscape]{article}

\usepackage{pgf,tikz}
%%%<
\usepackage{verbatim}
\usepackage[active,tightpage]{preview}
\PreviewEnvironment{tikzpicture}
\setlength\PreviewBorder{5pt}%
%%%>

\begin{comment}
:Title: 

\end{comment}

\usetikzlibrary{calc,arrows,positioning}
\usetikzlibrary{decorations.pathmorphing} % random steps (noise floor)

\usepackage{amsmath}
\usepackage[left=1cm,right=1cm]{geometry}
\pagestyle{empty}

\begin{document}

\begin{tikzpicture}[scale=10];
    \def\threshold{0.3}     % squelch threshold
    \def\noisefloor{0.15}   % noise floor level
    \def\ta{0.0}            %
    \def\tb{0.3}            %
    \def\rise{0.4}          % rising edge
    \def\tc{0.5}            %
    \def\td{1.3}            %
    \def\fall{1.4}          % falling edge
    \def\te{1.5}            %
    \def\timeout{1.8}       % timeout
    \def\tf{2.0}            %
    %\draw[step=2.5mm,help lines] (0,0) grid (2,1);

    % derived constants
    \def\cb{0.3*\tb   + 0.7*\rise}
    \def\cc{0.3*\rise + 0.7*\tc}
    \def\cd{0.7*\td   + 0.3*\fall}
    \def\ce{0.7*\fall + 0.3*\te}

    % axis 
    \draw[very thick, black,->] (-0.02, 0.0 ) -- (2,0) node[anchor=south] {{\it time}};
    \draw[very thick, black,->] ( 0.0, -0.02) -- (0,1) node[left=7pt,rotate=90] {{\it signal strength}};

    % draw curve
    \draw[very thick, black]
        (0,         \noisefloor) --
        (\ta,       \noisefloor) --
        (\tb,       \noisefloor) .. controls (\cb, \noisefloor) ..
        (\rise,     \threshold) .. controls (\cc, 1) ..
        (\tc,       1) --
        (\td,       1) .. controls (\cd, 1) ..
        (\fall,     \threshold) .. controls (\ce, \noisefloor) ..
        (\te,       \noisefloor) --
        (\timeout,  \noisefloor) --
        (\tf,       \noisefloor);

    % draw squelch threshold line
    \draw[thin, gray] (-0.02, \threshold) -- (2,\threshold)
        node[black,anchor=south] {{\it squelch threshold}};

    % draw noise floor
    \draw[thick, gray, decorate, decoration={random steps, segment length=4pt, amplitude=2pt}]
        (0, \noisefloor) -- (2,\noisefloor)
        node[black,anchor=south] {{\it noise floor}};

    % intersection lines
    \draw[thin,gray] (\rise,   \threshold)      -- (\rise,   -0.02);
    \draw[thin,gray] (\fall,   \threshold)      -- (\fall,   -0.02);
    \draw[thin,gray] (\timeout,\threshold)      -- (\timeout,-0.02);
    \draw[thin,gray] (\fall,   \threshold+0.02) -- (\fall,   \threshold+0.1);
    \draw[thin,gray] (\timeout,\threshold+0.02) -- (\timeout,\threshold+0.1)
        node[anchor=east,black] {{\it timeout}};
    \draw[thin,gray,->] (\fall,\threshold+0.07) -- (\timeout,\threshold+0.07);

    % intersection points
    \draw[fill=black] (\rise,   \threshold) circle (0.1pt);
    \draw[fill=black] (\fall,   \threshold) circle (0.1pt);
    \draw[fill=black] (\timeout,\threshold) circle (0.1pt);

    \draw (\rise,   \threshold) circle (0.25pt);
    \draw (\fall,   \threshold) circle (0.25pt);
    \draw (\timeout,\threshold) circle (0.25pt);

    % squelch code labels
    \def\q{-0.05};
    \path (0.00, \q) node[anchor=east] {{\it code:\,\,}}
          (0.00, \q) node {{\tt 0}}
          (0.05, \q) node {{\tt 0}}
          %(0.10, \q) node {{\tt 0}}
          (0.5*\rise, \q) node {$\cdots$}
          %(0.20, \q) node {{\tt 0}}
          (\rise-0.10, \q) node {{\tt 0}}
          (\rise-0.05, \q) node {{\tt 0}}
          (\rise,      \q) node {{\tt 1}}
          (\rise+0.05, \q) node {{\tt 2}}
          (\rise+0.10, \q) node {{\tt 2}}
          (\rise+0.15, \q) node {{\tt 2}}
          (0.5*\rise+0.5*\fall, \q) node {$\cdots$}
          (\fall-0.20, \q) node {$\cdots$}
          (\fall-0.15, \q) node {{\tt 2}}
          (\fall-0.10, \q) node {{\tt 2}}
          (\fall-0.05, \q) node {{\tt 2}}
          (\fall,      \q) node {{\tt 3}}
          (\fall+0.05, \q) node {{\tt 4}}
          (\fall+0.10, \q) node {{\tt 4}}
          (0.5*\fall+0.5*\timeout, \q) node {$\cdots$}
          (\timeout-0.10,   \q) node {{\tt 4}}
          (\timeout-0.05,   \q) node {{\tt 4}}
          (\timeout,        \q) node {{\tt 5}}
          (\timeout+0.05,   \q) node {{\tt 0}}
          (\timeout+0.10,   \q) node {{\tt 0}}
          (\timeout+0.15,   \q) node {{\tt 0}}
          (\timeout+0.20,   \q) node {$\cdots$};

    % circles around rising, falling edges
    \draw (\rise,   \q) circle (0.6pt);
    \draw (\fall,   \q) circle (0.6pt);
    \draw (\timeout,\q) circle (0.6pt);

\end{tikzpicture}

\end{document}

